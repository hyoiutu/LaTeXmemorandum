% ----- define document
\documentclass[12pt,uplatex]{jsarticle}
% ----- read package
%\usepackage{luatexja}
%\usepackage{luatexja-fontspec}
%\usepackage{lltjp-tascmac}
\usepackage{ascmac}
\usepackage{here}
\usepackage{xparse}
\usepackage{verbatim}
\usepackage{minted}
\usepackage[dvipdfmx]{hyperref}
\usepackage{pxjahyper}
% ----- define my Macros
\newminted[excode]{latex}{%
    linenos,
    frame=lines,
    numbersep=5pt,
    framesep=2mm
}
\NewDocumentEnvironment{myverbatim}{}{\verbatim}{\endverbatim}
% ----- title page
\title{\LaTeX 覚書}
\author{\vspace{-1zh}}\date{\vspace{-1zh}}
% ----- main pages
\begin{document}
  % ----- output title page\end{docu\end{document}ment}
  \maketitle
  % ----- output table of contents
  \tableofcontents
  % ----- new page
  \clearpage

  \section{\textbackslash author及び\textbackslash dateの省略方法}

  \begin{excode}
      \author{\textbackslash vspace{-1zh}}
      \date{\textbackslash vspace{-1zh}}
  \end{excode}
  と入力する.

  \section{長さの単位}

  \begin{table}[H]
  	\centering

  	\begin{tabular}{|c|c|} \hline
  		pt & ポイント(\verb|1pt| $\approx$ \verb|0.35mm|) \\ \hline
  		pc & パイカ(\verb|1pc| = \verb|12pt|) \\ \hline
  		bp & big point(\verb|1bp| $\approx$ \verb|1in|) \\ \hline
  		dd & didot point(\verb|1dd| $\approx$ \verb|1.07pt|) \\ \hline
  		mm & ミリメートル(\verb|1mm| $\approx$ \verb|2.85pt|) \\ \hline
  		cm & センチメートル(\verb|1cm| $\approx$ \verb|10mm|) \\ \hline
  		in & インチ(\verb|1in| = \verb|25.4mm|) \\ \hline
  		em & 現在有効な書体の文字Mの幅 \\ \hline
  		ex & 現在有効な書体の文字xの高さ \\ \hline
  		zw & 現在有効な全角漢字の幅 \\ \hline
  		zh & 現在有効な全角漢字の高さ \\ \hline
  	\end{tabular}
  \end{table}

  \section{新しい長さの定義}

  次のようにすることで新しい長さを定義できる.
  \begin{excode}
      \newlength{長さの名前}
      \setlength{長さの名前}{初期値}
  \end{excode}

  \section{特殊文字の出力}

  \begin{table}[H]
    \centering
  	\begin{tabular}{|c|c|} \hline
  		出力 & 入力 \\ \hline
  		\# & \verb|\#| \\ \hline
  		\$ & \verb|\$| \\ \hline
  		\% & \verb|\%| \\ \hline
  		\& & \verb|\&| \\ \hline
  		\textasciitilde & \verb|\textasciitilde| \\ \hline
  		\_ & \verb|\_| \\ \hline
  		\textasciicircum & \verb|\textasciicircum| \\ \hline
  		\textbackslash & \verb|\textbackslash|(数式モード内では\verb|\backslash|) \\ \hline
  		\{ & \verb|\{| \\ \hline
  		\} & \verb|\}| \\ \hline
  	\end{tabular}
  \end{table}

  \section{余白レイアウトの仕方}

  \begin{excode}
    \usepackage{\layout}
    \begin{document}
      \layout
      ...
    \end{document}
  \end{excode}
  でレイアウトを出力して余白を調整したい部分の名称を確認する.その後プリアンブル部で
  \begin{excode}
          \setlength{余白部分の名称}{長さ}
          \setlength{余白部分の名称}{長さ}
          $\vdots$
  \end{excode}
  のように調整する.長さの調整の時に例えばcm指定の場合はtruecmのように先頭にtrueをつけなければ実際の長さにならない.

  \section{複数の著者と所属をつける方法}

  \verb|\usepackage{authblk}|でパッケージを読み込んで
  \begin{excode}
      \author[1]{hoge}
      \author[2]{fuga}

      \affil[1]{piyo}
      \affil[2]{hogehoge}
  \end{excode}

  \section{段落の最初に字下げしない方法}

  字下げしたくない段落で\verb|\noindent|と書く.すべての段落で字下げを行わないようにするにはプリアンブル部に\verb|\parindent = 0pt|と書く.

  \section{mintedで載せたソースコードがページをまたぐ場合}

  通常,mintedで載せたソースコードはページを跨いでも適切に処置してくれるが以下のいずれかの条件を満たす時,ソースコードがページをはみ出す.
  \begin{itemize}
      \item listing環境を使った時
      \item mintedにbgcolorオプションを付加した時
  \end{itemize}
  mintedで載せたソースコードが1ページに収まらない場合はmdfaramed環境を使う.次のような構造で使う.
  \begin{excode}
      \usepackage{mdframed}
      \usepackage[cache=false]{minted}
      \begin{document}
      \begin{mdframed}
          \begin{minted}{c}
          ...
          \end{minted}
      \end{mdframed}
      \end{document}
  \end{excode}
  mdframed環境はデフォルトでボックスで囲まれているのでmintedの上下左右の線とかぶる場合がある.その場合は次のようなオプションをつけるとよい.
  \begin{excode}
     \begin{mdframed}[
         topline=false,
         bottomline=false,
         leftline=false,
         rightline=false]
     \end{mdframed}
  \end{excode}
  mdframed環境を使う場合はlisting環境が使えずキャプションを通常どおりにつけることができない.その為,captionof環境を使う.使い方としては次の通りである.
  \begin{excode}
      \usepackage{caption}
      ...
      \captionof{*}{*}
      ...
  \end{excode}
  captionof環境はcaptionパッケージの中にあるので読み込む.captionof環境の第一引数にはキャプションの種類を記述する.メジャーなのは次の3つである.
  \begin{table}[htbp]
      \begin{tabular}{|c|c|} \hline
          表示 & 引数に取る文字列 \\ \hline
          図○ & figure \\ \hline
          表○ & table \\ \hline
          Listing○ & listing \\ \hline
      \end{tabular}
  \end{table}
  第二引数にはキャプションにする文章を入力する.

  \section{目次の作り方及び目次に表示する節のレベルを設定する}

  \begin{excode}
      \tableofcontents
  \end{excode}
  表示はできるが\verb|\subsubsection|などの小々節は目次に載らない.こういう時は次のコマンドをプリアンブル部に記述する.
  \begin{excode}
      \setcounter{tocdepth}{3}
  \end{excode}
  tocdepthというのは目次に載せる節の深さを表しておりデフォルトで2である.よって\verb|\subsubsection|以下のレベルは目次に載らない.なので\verb|\setcounter|コマンドで変数に入っている値を変更する.値を増やせば細かく目次に載り,逆に値を減らせば大雑把な目次になる.
  \section{目次にありもしない項目を加える}
  \verb|\addcontentsline{file}{type}{text}|でsectionで生成している節以外に目次に項目を増やせる. \\
  \begin{table}[H]
      \caption{引数fileの取りうる値}
      \begin{tabular}{c|c}
          toc & partやsection,subsectionなどで生成される一般的な目次 \\ \hline
          figure & 図の一覧目次 \\ \hline
          table & 表の一覧目次 \\
      \end{tabular}
  \end{table}
  \noindent
  type \\
  sectionなら本来の目次のsection並の字下げを行う.subsectionも然り.\\
  text \\
  目次に追加したい項目を書く
  \section{条件分岐を行うマクロなどifthenパッケージ}
  条件分岐,命題,繰り返しなどほとんどプログラミング言語に近づけたような制御マクロが詰まっているifthenパッケージ\verb|\usepackage{ifthen}|とすれば使える.
  詳しい使い方はここのURLから
  \url{http://qiita.com/zr_tex8r/items/71ae46c9c4e8cb575073#_reference-9aee38fe27ee3682b28a}
  \section{\LaTeX 3でデフォルトになるマクロ定義マクロが入ったxparseパッケージ}

  \LaTeX 3のチームが開発したオプション引数が複数指定できたりと本来のマクロ定義コマンドよりも自由度が高いマクロ定義コマンドが入ったxparseパッケージ.使い方は以下から
  \url{http://qiita.com/zr_tex8r/items/50168ad7087516c3e139}
  \section{マクロ内でverbatim環境を使う方法}
  \verb|\usepackage{fancybox}|でfancyboxパッケージを読み込む.以下のように使う.
  \begin{excode}
      \NewDocumentEnvironment{myverbatim}{%
      \VerbatimEnvironment
      \begin{Verbatim}
  }
  {
      \end{Verbatim}
  }%
  \end{excode}
  このように最初に\verb|\VerbatimEnvironment|を指定するのがミソである.

  \section{ラベルからページ番号を参照する方法}

  \verb|\label{}|などとすると\verb|\ref{}|で表,図,数式などの番号を参照することができる.しかし,\verb|\pageref{}|とすると,参照元のページ番号を出力することができる
  \section{inputした\TeX ファイルなどを拡大・縮小する方法}
  graphcsパッケージを読み込み次のように記述する.
  \begin{excode}
      \usepackage{graphcs}
      \begin{document}
      ...
      \scalebox{倍率}{\input{ファイル名}}
      ...
      \end{document}
  \end{excode}

  \section{組版におけるクォートのルールについて}

  \LaTeX などの組版処理システムにおいてはクォートで文章を囲む時にルールがある.
  シングルクォートで囲む時は始まりは\verb|`|(バッククォート)で始まり, \verb|'|(シングルクォート)で終わる.ダブルクォートで囲む時は始まりは\verb|``|(バッククォート2つ)始まり,\verb|''|(シングルクォート2つ)で終わる.

  \section{ページを跨ぐ表の作り方}

  通常の表はtabular環境とtable環境を使って作るが1ページに収まらないような大きな表は通常,ページを跨ぐことができない.この場合はlongtableパッケージを読み込みlongtable環境を使う.
  \begin{excode}
      \usepackage{longtable}
      \begin{document}
      \begin{longtable}[表の位置指定]{行数指定}
          ...
      \end{longtable}
  \end{excode}
  \begin{table}[htbp]
      \centering
      \caption{表の位置指定に使えるオプション}
      \begin{tabular}{c|c}
          オプション & 効果 \\ \hline
          l & 左詰め \\ \hline
          c & 中央揃え \\ \hline
          r & 右詰め \\
      \end{tabular}
  \end{table}
  行数指定は通常のtabular環境と一緒である.

  \section{表の各セルの高さを調整する方法}

  全てセルを高くするには次のようにする.
  \begin{excode}
      \renewcommand{\arraystretch{拡大率}}
  \end{excode}
  一部のセルだけ高くするにはruleコマンドを使う.
  \begin{excode}
  \newlength{¥myheight}
  \setlength{¥myheight}{1cm}
  \newlength{¥myheighta}
  \setlength{¥myheighta}{2cm}

  \begin{tabular}{|c|c|}
  \hline
  default & default \\
  \hline
  \rule{0cm}{¥myheight} 1cmの高さ & ああ \\
  \hline
  \rule{0cm}{¥myheighta} 2cmの高さ & いい \\
  \hline
  default & default \\
  \hline
  \end{tabular}
  \end{excode}
  このコードの実行結果は次のようになる.
  \newlength{\myheight}
\setlength{\myheight}{1cm}
\newlength{\myheighta}
\setlength{\myheighta}{2cm}


\begin{table}[htbp]
  \centering
  \begin{tabular}{|c|c|}\hline
    default & default \\\hline
    \rule{0cm}{\myheight} 1cmの高さ & ああ \\\hline
    \rule{0cm}{\myheighta} 2cmの高さ & いい \\\hline
    default & default \\\hline
  \end{tabular}
\end{table}


  \section{美しいボックスを描けるtcolorbox環境}

  使うにはtcolorboxパッケージを読み込む.例などは \url{http://ftp.jaist.ac.jp/pub/CTAN/macros/latex/contrib/tcolorbox/tcolorbox.pdf}に載っている.

  \section{tcolorbox環境で美しい表を作る}

  作るにはtcolorbox,array,tabularxのパッケージを読み込む.tcolorbox環境のオプションで次のように指定することで表を作ることができる.
  \begin{excode}
      \begin{tcolorbox}[tabularx*={...]
              ...
      \end{tcolorbox}
  \end{excode}
  オプション内の...に続くのは実際にtabularxで使える命令やオプションである.
  例として,同じディレクトリにあるtcolortable.texを参照するとよい.

  \makeatletter
  \section{マクロで@を使う意味}

  \verb|@|はカテゴリーコードで12に割り振られているのでその他の文字として扱われる。つまり、普通の\TeX の文書を作っていて出てくる文字ではないという扱いである。そのため、マクロなどの名前に\verb|@|を使うのは重複を防止するためである。\verb|@|のカテゴリーコードを変更するには\verb|\makeatletter|と\verb|\makeatother|で囲む。
  \makeatother

  \section{マクロの行末に\%を入れる理由}
  \TeX のスタイルファイルなどを見てみると行末に\%だけが記述されていることがある.これは\TeX の空白のルール上の対策である.
  \TeX の空白が無視されるかどうかは次のルールに従う.
  \begin{enumerate}
    \item 行頭と行末にある空白文字・タブは全て無視される。
    \item コメントの直後の改行文字は無視される。%
    \label{item:ignoreComNexInd}
    \item 空行(連続する改行文字)は『改段落』を表す。
    \item (空行以外の)改行文字は空白文字になる。 \\
          (ただし pTeX 系で、行末が和文文字の場合は改行文字は無視される。)%
          \label{item:convIndToSpa}
    \item 1つ以上の空白文字・タブの並びは単一の空白文字と見做される。
    \item 制御語(名前が英字からなる制御綴)の直後の空白文字は無視される。
    \item それ以外の空白文字が『意味のある空白文字』となる。
  \end{enumerate}
  上記のルールのうちの\ref{item:convIndToSpa}番の影響で空白が入ってしまう為,
  ルールの\ref{item:ignoreComNexInd}番に従って対策をする.よって\%を入れる.

  \section{sectionなどのフォントを変える方法}
  sectionのフォントはデフォルトで欧文:サンセリフ体,和文:ゴシック体である.
  これを変えるには次の\verb|\headfont|を再定義する.
  \begin{excode}
    \renewcommand{\headfont}{<和文のフォント><欧文のフォント>}
  \end{excode}
  \section{文章全体のフォントを変える方法}
  文章全体の欧文及び和文のフォントを変えるには次のコマンドを再定義する.
  \begin{excode}
    \renewcommand{\kanjifamilydefault}{\gtdefault}
    \renewcommand{\familydefault}{\sfdefault}
  \end{excode}
  マクロ名はフォント名+familyではなく,フォント名+defaultなので注意
  \section{\TeX で標準入力及び標準出力を行う方法}
  次のコマンドで標準入力を行う.
  \begin{excode}
    \newcount \tmp
    \read-1 \tmp
  \end{excode}
  次のコマンドで標準出力を行う
  \begin{excode}
    \message{何か入力してください}
  \end{excode}
  \section{beamerでノートを使う方法}
  PowerPointのような発表者用のノートはbeamerでは次のように入力する.
  \begin{excode}
    \note{...}
  \end{excode}
  また,ノートだけを表示したい場合はプリアンブルで
  \begin{excode}
    \documentclass[notes=only]{beamer}
  \end{excode}
  という風にする.
  また,どちらも表示したい場合は
  \begin{excode}
    \documentclass[notes]{beamer}
  \end{excode}
  とする.
  \section{mintedでアセンブラのソースを載せる時の注意}
  アセンブラといえどアセンブラの種類はCPUの種類だけあると言っていいほど書き方があるので
  mintedの設定を間違えないように気をつける.
  minted環境の第一引数にasmと持ってきがちだがこれはX86のアセンブラなのでintelのアセンブラとは
  違うものとなっている.アセンブラごとの引数の違いはURL先のそれぞれのShort Nameを参照すること.\\
  \url{http://pygments.org/docs/lexers/} \\
  mintedはpymentsをベースにシンタックスハイライトしている.pytmentsは字句解析(lexer)を行う
  ツールなので引数のとり方によっては字句解析の結果,過多なシンタックスハイライトがついてる場合が
  あるので邪魔だと思う場合は名前にlexerとついていないものを選ぶこと.因みにasmは字句解析を行う.

  \section{\TeX で言語処理する第一歩}
  \subsection{\TeX で文字列置換}
  \TeX で文字列置換を行うには\verb|\@tfor\ch:=hoge\do{}|を使う.
  これはhogeという文字列を\verb|\@tfor|で一文字ずつ見ていくというマクロである.
  \verb|\@tfor|で回している間のカウンタ変数は\verb|\ch|である.
  次に例を挙げる.次の例はxという文字を\textbackslash xに置き換えるマクロである.
  \begin{excode}
    \@tfor\ch:=#1\do{\ifthenelse{\equal{\ch}{x}}{\textbackslash x}{\ch}}%
  \end{excode}

  \section{ドットの長さで後尾を揃えてくれる\textbackslash dotfill}
  \verb|\dotfill|はドットの前の文字列の長さに合わせてドットを出力して後尾の位置を合わせてくれる
  マクロである.以下,例.
  \begin{excode}
     おこ \dotfill レベル1 \\
     激おこ \dotfill レベル2 \\
     激おこぷんぷん丸 \dotfill レベル3 \\
     ムカ着火ファイア \dotfill レベル4 \\
     カム着火インフェルノォォォオオオウ \dotfill レベル5 \\
     げきオコスティックファイナリアリティぷんぷんドリーム \dotfill レベル6
  \end{excode}
  \begin{itembox}[l]{出力結果}
    おこ \dotfill レベル1 \\
    激おこ \dotfill レベル2 \\
    激おこぷんぷん丸 \dotfill レベル3 \\
    ムカ着火ファイア \dotfill レベル4 \\
    カム着火インフェルノォォォオオオウ \dotfill レベル5 \\
    げきオコスティックファイナリアリティぷんぷんドリーム \dotfill レベル6
  \end{itembox}

  \section{verbatim環境内で自動改行を有効にするspverbatim環境}
  verbatim環境は入力したテキストをそのまま出力する為,自動改行されない.
  spverbatim環境を使うと自動改行を有効にする.
\end{document}
